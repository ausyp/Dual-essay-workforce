
\documentclass[a4paper,man,british]{apa6}
\usepackage[british]{babel}
\usepackage[utf8]{inputenc}
\usepackage{csquotes}
\usepackage[hidelinks]{hyperref}
\usepackage[style=apa]{biblatex}
\DeclareLanguageMapping{british}{british-apa}

% maps apacite commands to biblatex commands
\let \citeNP \cite
\let \tcite \textcite
\let \cite \parencite


\addbibresource{zotero_references.bib}
\addbibresource{bibliography.bib}

\title{Workforce challenges with clients with dual diagnosis}
\shorttitle{dual diagnosis and Workforce challenges}
\author{Austin Paul}
\affiliation{RMIT UNIVERSITY \\ s3634517}




\begin{document}

\maketitle

\section{Introduction} %100

\section{Workforce challenges for training}

The Australian Institute of Health and Welfare has identified culturally adapted training as challenge faced by health care. This is more so in the case of Indigenous Australians who have higher rates of mental illness \cite{hinton_evaluation_2012}. The lack of a culturally adapted  training significantly reduce the effectiveness of the service or in some circumstances detrimental to the clients \cite{drew_issues_2010}. For example the Aboriginal concept of time is not exclusively linear like in a western perception of time but relativistic in nature according to the importance of an event in time, this makes a whole host of standard assessments that professional's are trained to do limited in its applicability \cite{janca_aboriginal_2003}. Much of the assessments in mental health globally are based on a western framework and this has significant effects when working with different indigenous groups, as mentioned by \textcite{drew_issues_2010} even fairly recent assessment guideline published in Australia fails to address or even mention the relevant  cultural factors that might affect the assessment. As identified above there is significant reflection of lack of culturally adapted training for workforce in mental health.


\section{Conclusion}%100


%\printbibliography

\end{document}
