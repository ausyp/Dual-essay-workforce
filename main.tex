
\documentclass[a4paper,man,british]{apa6}
\usepackage[british]{babel}
\usepackage[utf8]{inputenc}
\usepackage{csquotes}
\usepackage[hidelinks]{hyperref}
\usepackage[style=apa]{biblatex}
\DeclareLanguageMapping{british}{british-apa}

% maps apacite commands to biblatex commands
\let \citeNP \cite
\let \tcite \textcite
\let \cite \parencite


\addbibresource{zotero_references.bib}
\addbibresource{bibliography.bib}

\title{Workforce challenges with clients with dual diagnosis}
\shorttitle{dual diagnosis and Workforce challenges}
\author{Austin Paul}
\affiliation{RMIT UNIVERSITY \\ s3634517}




\begin{document}

\maketitle

\section{Introduction} %100

"At the heart of each and every health system, the workforce is central to advancing health" is a quote from \tcite{geneva2006world} World Health Report. As the report acknowledges health care workforce is at the core of every advancement in health care and the challenges faced in workforce training are of significant importance. This paper aims to discuss the workforce challenges for training in mental health and recommendations for implementation of strategies to mitigate them.

\section{Workforce challenges for training}

The Australian Institute of Health and Welfare (2018) has identified culturally adapted training as a challenge faced by health care. This is more so in the case of Indigenous Australians who have higher rates of mental illness \cite{hinton_evaluation_2012}. The lack of a culturally adapted  training significantly reduce the effectiveness of the service or in some circumstances detrimental to the clients \cite{drew_issues_2010}. For example the Aboriginal concept of time is not exclusively linear like in a western perception of time but relativistic in nature according to the importance of an event in time, this makes a whole host of standard assessments that professional's are trained to do limited in its applicability \cite{janca_aboriginal_2003}. Much of the assessments in mental health globally are based on a western framework and this has significant effects when working with different indigenous groups, as mentioned by \textcite{drew_issues_2010} even fairly recent assessment guideline published in Australia fails to address or even mention the relevant  cultural factors that might affect the assessment. As identified above there is significant reflection of lack of culturally adapted training for workforce in mental health.

\textcite{au-2014} identified location and lack of funding as a challenge in training workforce especially in the regional areas. It cited that inconsistencies in funding cycle pattern does not incentives specialised training due to their temporary nature and low likelihood of professionals staying in rural communities. Research done by \tcite{fraser_changing_2005} has shown that rural location negatively effect mental health of the residents of the that region and is a strong predictor for for poor mental health service availability. The research has also shown that areas with declining population has had the pre-existing infrastructure in the region closed or services withdrawn. Another factor identified by the \tcite{au-2014} is the lack of infrastructure to provide training to existing workforce in remote areas. All of this can be attributed to a lack funding as the mental health services only accounts for less than five percent of the governments healthcare budget at a time when 20\% of Australians are estimated to have mild to severe mental illness\cite{australian_government_department_of_health_australian_2018,australian_government_department_of_health_national_2013}.
\\
A challenge faced in training of nursing workforce for mental health is their limited exposure to mental illness in undergraduate studies as shown in a study conducted by \tcite{wynaden_are_2008}. This study identified that the nursing curricula does not address necessary aspects of mental illness. The results of this study showed that majority of graduate nurses perceived themselves to be unprepared to work with mentally ill. This leads to the next challenge faced in training of mental health nursing workforce; recruitment and retention as identified by \tcite{au-2014,cleary_promoting_2005}. The significant challenges faced in recruitment and retention is a barrier to providing training.
%The transition to practice programme implemented at CSAMHS has been positively evaluated by the participant graduates, which suggests it may provide a useful strategy for the recruitment and retention of new graduates. Further development is required to ensure that graduates have greater access to preceptorship within the clinical environment.
There is a lot of research exploring the benefits of service user involvement in training of mental health workforce especially nursing students and a lot this study's has shown that it is in fact beneficial \cite{gilbert_wounded_2012,simons_socially_2007,weinstein_mental_2010}.  Involvement of service users in training can help students understand mental health illness better and influence thier attitude towards them in a positive manner. Research done by \tcite{happell_mental_2015} in Australia has shown that even-though there is strong support for including people with lived experience, most universities had limited arrangement for service user engagement and was mostly informal exercises.

\section{Recommendations for implementation of effective
strategies}

Developing culturally adapted training is required to achieve a transition from historical attitudes and beliefs towards indigenous people; to its credit The Australian government did declare this as the first priority in the fourth national mental health plan \cite{australian_government_department_of_health_and_fourth_2009}. But we still have a long way to go as pointed out by \tcite{nagel_yarning_2012}, there are still toxic combination of social factors that are hindering the process of closing the gap in indigenous health. When \tcite{dingwall_evaluation_2015} conducted a study in which participants were trained with a culturally adapted mental health intervention course; the participants showed significant improvements. There is a need to develop more culturally adapted training courses which is more inclusive of everyone.\
As mentioned earlier, mental health services funding only accounts for less than five percent of the Australian governments health care budget when the incidence mental illness is about 20 percent. It is also noticed that sometimes mental health funding is not distributed proportional to the burden caused by the illness in communities \cite{milner_suicide_2018}. The National Health and Medical Research Council Australia on average has allocated less than nine percent of its funding to mental health research, very low compared to the burden of disease \cite{batterham_nhmrc_2016}. Without sufficient research, evidence based training will stagnant so there is a need for more funding.
There is need to further incorporate mental health studies into undergraduate degrees like nursing to enhance students understanding the illnesses there is also need to improve upon current mental health placements to increase students preparedness to work in mental health and recive further training \cite{wynaden_are_2008}.

More involvement service users are needed to understand the experiences of the people who use mental health services and improve upon current training \cite{happell_mental_2015}.Studies has shown that including service users with lived experience have positive out come for nursing students and academics \cite{gilbert_wounded_2012,rush_mental_2008,morgan_perceptions_2009}. There is a need to further improve the involvement people with lived experience in training to increase the effectiveness such training.  
%Including service users in nurse education programmes has been reported to have positive outcomes for students, service users and nurse academics

\section{Conclusion}%100

This paper attempted to explore the workforce for training and make recommendations for change. It was consistently found that there are aspects and parts of training process that face significant challenges due to lack funding, lack of interest from training providers, lack of infrastructure, inability to involve service users and lack of literature exploring different training methods and its benefits. There is a need to draw more public attention to mental health issues and show that research in the field can help find solutions and help reduce the cost of the these illnesses. Advancing the training provided to health care professionals can and are shown to have positive outcomes for individual's and communities who receive their service and therefore efforts must be made to that end.

\printbibliography

\end{document}
